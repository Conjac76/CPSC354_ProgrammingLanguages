\documentclass{article}
\usepackage{amsmath}
\usepackage{forest}
\usepackage{tikz} 
\usepackage{tikz-cd} % For diagrams
 \usepackage{quiver}
\usetikzlibrary{automata, positioning, arrows, cd} 

\usepackage{amsthm}
\usepackage{amsfonts}
\usepackage{amsmath}
\usepackage{amssymb}
\usepackage{fullpage}
\usepackage{color}
\usepackage{parskip}
\usepackage{hyperref}
  \hypersetup{
    colorlinks = true,
    urlcolor = blue,       % color of external links using \href
    linkcolor= blue,       % color of internal links 
    citecolor= blue,       % color of links to bibliography
    filecolor= blue,        % color of file links
    }

\usepackage{listings}

\definecolor{dkgreen}{rgb}{0,0.6,0}
\definecolor{gray}{rgb}{0.5,0.5,0.5}
\definecolor{mauve}{rgb}{0.58,0,0.82}

\lstset{frame=tb,
  language=haskell,
  aboveskip=3mm,
  belowskip=3mm,
  showstringspaces=false,
  columns=flexible,
  basicstyle={\small\ttfamily},
  numbers=none,
  numberstyle=\tiny\color{gray},
  keywordstyle=\color{blue},
  commentstyle=\color{dkgreen},
  stringstyle=\color{mauve},
  breaklines=true,
  breakatwhitespace=true,
  tabsize=3,
  escapeinside={(*@}{@*)} % Custom escape settings to handle special chars
}

\newtheoremstyle{theorem}
  {\topsep}   % ABOVESPACE
  {\topsep}   % BELOWSPACE
  {\itshape\/}  % BODYFONT
  {0pt}       % INDENT (empty value is the same as 0pt)
  {\bfseries} % HEADFONT
  {.}         % HEADPUNCT
  {5pt plus 1pt minus 1pt} % HEADSPACE
  {}          % CUSTOM-HEAD-SPEC
\theoremstyle{theorem} 
   \newtheorem{theorem}{Theorem}[section]
   \newtheorem{corollary}[theorem]{Corollary}
   \newtheorem{lemma}[theorem]{Lemma}
   \newtheorem{proposition}[theorem]{Proposition}
\theoremstyle{definition}
   \newtheorem{definition}[theorem]{Definition}
   \newtheorem{example}[theorem]{Example}
\theoremstyle{remark}    
  \newtheorem{remark}[theorem]{Remark}

\title{CPSC-354 Report}
\author{Connor Jacobs \\ Chapman University}

\date{\today} 

\begin{document}

\maketitle

\begin{abstract}
This report will contain a summary of my learning progress throughout the course, so far including: lean proofs
\end{abstract}

\setcounter{tocdepth}{3}
\tableofcontents

\section{Introduction}\label{intro}

This report will document my learning through the course. The content will be structured week by week, with sections on mathematical notes, homework solutions, and personal reflections on the topics discussed.

\section{Week by Week}\label{homework}

\subsection{Week 1}

\subsubsection*{Mathematical Proof}

Proving that $37x + q = 37x + q$ demonstrates the reflexivity property of equality. Reflexivity states that any mathematical expression is equal to itself. In this case, the expression $37x + q$ is compared to itself, and it is immediately clear by the reflexivity of equality that this statement is true.

\subsubsection*{Proof in Lean}

In Lean, the theorem $37x + q = 37x + q$ is proven using the `rfl` tactic. The `rfl` tactic in Lean stands for "reflexivity" and handles proofs of the form $X = X$. When `rfl` is executed, Lean verifies that both sides of the equation are equal and the proof is complete.

The command to execute in Lean is simply:
\begin{verbatim}
rfl
\end{verbatim}

\subsubsection*{Connection Between Lean and Mathematics}

In mathematics, we rely on the axiom of reflexivity to assert that $37x + q = 37x + q$. Likewise, in Lean, the `rfl` tactic automates this process by invoking the same principle. The use of `rfl` in Lean serves as a direct representation of reflexivity in mathematical logic, providing an automated and formalized way to conclude that an expression is equal to itself.

Thus, both in Lean and in traditional mathematics, the proof of $37x + q = 37x + q$ is an application of the same fundamental concept: the reflexivity of equality.

\subsubsection*{Answers to Levels 5-8}

\subsubsection*{Level 5}

\begin{verbatim}
rw [add_zero]
rw [add_zero]
rfl
\end{verbatim}

\subsubsection*{Level 6}

\begin{verbatim}
rw [add_zero c]
rw [add_zero b]
rfl
\end{verbatim}

\subsubsection*{Level 7}

\begin{verbatim}
rw [one_eq_succ_zero]
rw [add_succ]
rw [add_zero]
rfl
\end{verbatim}

\subsubsection*{Level 8}

\begin{verbatim}
rw [two_eq_succ_one]
rw [one_eq_succ_zero]
rw [four_eq_succ_three]
rw [three_eq_succ_two]
rw [two_eq_succ_one]
rw [one_eq_succ_zero]
rw [add_succ]
rw [add_succ]
rw [add_zero]
rfl
\end{verbatim}

\subsection{Week 2}

\subsubsection*{Mathematical Proofs and Lean Implementation}

\textbf{Theorem (Associativity of Addition):} For all natural numbers $a$, $b$, and $c$, the addition operation is associative: $(a + b) + c = a + (b + c)$.

\textit{Proof:} By induction on $c$:

\begin{itemize}
    \item \textbf{Base Case:} When $c = 0$, we have:
    \[
    (a + b) + 0 = a + (b + 0)
    \]
    Using the property addition with zero, this simplifies to:
    \[
    a + b = a + b
    \]
    This equality holds by the reflexivity of equality

    \item \textbf{Inductive Step:} Assume the hypothesis holds for some $d$
    \[
    (a + b) + d = a + (b + d)
    \]
    We must prove $c = \text{succ}(d)$:
    \[
    (a + b) + \text{succ}(d) = a + (b + \text{succ}(d))
    \]
    this can be rewritten as:
    \[
    \text{succ}((a + b) + d) = \text{succ}(a + (b + d))
    \]
    By the inductive hypothesis, we know $(a + b) + d = a + (b + d)$, so:
    \[
    \text{succ}((a + b) + d) = \text{succ}(a + (b + d))
    \]
    This completes the proof of associativity.
\end{itemize}

\textbf{Connection to Lean Proof of Associativity of Addition:}

The Lean proof is the same structure of the mathematical proof:

\begin{verbatim}
induction c with d hd
-- Base case
rw [add_zero]  -- Simplifies (a + b) + 0 to a + b using add_zero
rw [add_zero]  -- Simplifies a + (b + 0) to a + b using add_zero
rfl            -- Reflexivity: confirms a + b = a + b
-- Inductive step
rw [add_succ]  -- Rewrites (a + b) + succ d as succ ((a + b) + d)
rw [hd]        -- Applies the inductive hypothesis: (a + b) + d = a + (b + d)
rw [add_succ]  -- Rewrites a + (b + succ d) as succ (a + (b + d))
rw [add_succ]  -- Confirms that the expressions on both sides match
rfl            -- Reflexivity: confirms succ (a + (b + d)) = succ (a + (b + d))
\end{verbatim}

\textbf{Explanation:}

\begin{itemize}
  \item \textbf{Base Case in Lean:}
  \begin{itemize}
      \item \texttt{rw [add\_zero]} is used twice to simplify both sides of the equation to $a + b$, directly corresponding to the mathematical step where $a + b = a + b$ is shown by reflexivity.
      \item \texttt{rfl} is then used to verify
  \end{itemize}

  \item \textbf{Inductive Step in Lean:}
  \begin{itemize}
      \item \texttt{rw [add\_succ]} rewrites the expression using the definition of addition with the successor, corresponding to the mathematical step where we transition from $(a + b) + \text{succ}(d)$ to $\text{succ}((a + b) + d)$.
      \item \texttt{rw [hd]} applies the inductive hypothesis, mirroring the assumption that $(a + b) + d = a + (b + d)$.
  \end{itemize}
\end{itemize}

\subsubsection*{Week 2 questions}

\textbf{Question:} Can Lean be used to verify complex operations in critical systems such as financial transactions or autonomous vehicles? What are the limitations of its use in such a high stakes environment?

\subsection{Week 3}

\textbf{Discord Name:} connorjacobs

\textbf{Discord Posting:} In my research, I took a look at how advancements in programming languages have balanced performance, safety, and ease of use, focusing on Rust and TypeScript. Rust’s ownership model enforces memory safety by controlling how data is accessed and managed, preventing issues like memory leaks and race conditions without garbage collection. In contrast, TypeScript enhances JavaScript by introducing type checking, catching type errors early, and improving overall code reliability.

\textbf{Link to ReadMe:} \\
\href{https://github.com/Conjac76/WeekThree354/blob/main/README.md}{https://github.com/Conjac76/WeekThree354/blob/main/README.md}

\textbf{Links to two reviews I voted for:} \\
\href{https://github.com/tannerplatt/TannerHW3_354/blob/main/README.md}{https://github.com/tannerplatt/TannerHW3\_354/blob/main/README.md} \\
\href{https://github.com/tyedwards37/Using-LLM-for-Literature-Review}{https://github.com/tyedwards37/Using-LLM-for-Literature-Review}


\subsection{Week 4}

\subsection*{Derivation Trees}

Below are the derivation trees for the given strings using the provided context-free grammar:

\subsubsection*{1. Derivation Tree for \(2+1\)}

\begin{forest}
  [Exp
    [Exp ['2']]
    [\texttt{'+'}]
    [Exp1
      [Exp2 ['1']]
    ]
  ]
\end{forest}

\subsubsection*{2. Derivation Tree for \(1+2*3\)}

\begin{forest}
  [Exp
    [Exp ['1']]
    [\texttt{'+'}]
    [Exp1
      [Exp1 
        [Exp2 ['2']]
      ]
      [\texttt{'*'}]
      [Exp2 ['3']]
    ]
  ]
\end{forest}

\subsubsection*{3. Derivation Tree for \(1+(2*3)\)}

\begin{forest}
  [Exp
    [Exp ['1']]
    [\texttt{'+'}]
    [Exp1
      [Exp2
        [\texttt{'('}]
        [Exp
          [Exp1
            [Exp1
              [Exp2 ['2']]
            ]
            [\texttt{'*'}]
            [Exp2 ['3']]
          ]
        ]
        [\texttt{')'}]
      ]
    ]
  ]
\end{forest}

\subsubsection*{4. Derivation Tree for \((1+2)*3\)}

\begin{forest}
  [Exp
    [Exp1
      [Exp1 
        [Exp2
          [\texttt{'('}]
          [Exp
            [Exp1
              [Exp
                [Exp1
                  [Exp2 ['1']]
                ]
              ]
              [\texttt{'+'}]
              [Exp1
                [Exp2 ['2']]
              ]
            ]
          ]
          [\texttt{')'}]
        ]
      ]
      [\texttt{'*'}]
      [Exp2 ['3']]
    ]
  ]
\end{forest}

\subsubsection*{5. Derivation Tree for \(1+2*3+4*5+6\)}

\begin{forest}
  [Exp
    [Exp
      [Exp 
        [Exp1
          [Exp2 ['1']]
        ]
      ]
      [\texttt{'+'}]
      [Exp1
        [Exp1
          [Exp1 
            [Exp2 ['2']]
          ]
          [\texttt{'*'}]
          [Exp2 ['3']]
        ]
      ]
    ]
    [\texttt{'+'}]
    [Exp1
      [Exp1
        [Exp1 
          [Exp2 ['4']]
        ]
        [\texttt{'*'}]
        [Exp2 ['5']]
      ]
      [\texttt{'+'}]
      [Exp1
        [Exp2 ['6']]
      ]
    ]
  ]
\end{forest}

\subsection{Week 5}

\section*{Solutions to Levels 1-8 (Lean Syntax)}

\subsection*{Level 1}
\texttt{exact todo\_list}

\subsection*{Level 2}
\texttt{exact and\_intro p s}

\subsection*{Level 3}
\texttt{exact and\_intro (and\_intro a i) (and\_intro o u)}

\subsection*{Level 4}
\begin{verbatim}
have p := and_left vm
have p := vm.left
exact p
\end{verbatim}

\subsection*{Level 5}
\texttt{exact h.right}

\subsection*{Level 6}
\texttt{exact and\_intro (and\_left h1) (and\_right h2)}

\subsection*{Level 7}
\begin{verbatim}
have h1 := h.left
have h2 := h1.right
have h3 := h2.left
have h4 := h3.left
have h5 := h4.right
exact h5
\end{verbatim}

\subsection*{Level 8}
\begin{verbatim}
have h1 := h.left
have h2 := h1.left
have h3 := h1.right
have h4 := h.right.right.left
have h5 := h4.left
exact and_intro h3 (and_intro h5 (and_intro h2.left h2.right))
\end{verbatim}

\subsection*{Level 8: Proof in Mathematical Logic}

If $P \land (Q \land R) \land ((S \land T) \land (U \land V))$, then $P \land (S \land U \land V)$.

\begin{align*}
(1) &\quad P \land (Q \land R) \land ((S \land T) \land (U \land V)) \quad & \text{assumption} \\
(2) &\quad P \land (Q \land R) \quad & \text{and\_left (1)} \\
(3) &\quad P \quad & \text{and\_left (2)} \\
(4) &\quad (S \land T) \land (U \land V) \quad & \text{and\_right (1)} \\
(5) &\quad S \land T \quad & \text{and\_left (4)} \\
(6) &\quad U \land V \quad & \text{and\_right (4)} \\
(7) &\quad S \quad & \text{and\_left (5)} \\
(8) &\quad U \quad & \text{and\_left (6)} \\
(9) &\quad V \quad & \text{and\_right (6)} \\
(10) &\quad P \land (S \land U \land V) \quad & \text{and\_intro (3) (7) (8) (9)}
\end{align*}


\subsection{Week 6}


\subsubsection*{Level 1}
\begin{verbatim}
have c : C := bakery_service p
exact c
\end{verbatim}

\subsubsection*{Level 2}
\begin{verbatim}
exact /lambda h : C -> h
\end{verbatim}

\subsubsection*{Level 3}
\begin{verbatim}
exact /lamba h : I ^ S -> ⟨h.right, h.left⟩
\end{verbatim}

\subsubsection*{Level 4}
\begin{verbatim}
exact /lamba c : C -> h2 (h1 c)
\end{verbatim}

\subsubsection*{Level 5}
\begin{verbatim}
have q : Q := h1 p
have t : T := h3 q
have u : U := h5 t
exact u
\end{verbatim}

\subsubsection*{Level 6}
\begin{verbatim}
exact fun c : C => fun d : D => h ⟨c, d⟩
\end{verbatim}

\subsubsection*{Level 7}
\begin{verbatim}
exact fun cd : C ^ D => h cd.left cd.right
\end{verbatim}

\subsubsection*{Level 8}
\begin{verbatim}
exact fun s : S => ⟨h.left s, h.right s⟩
\end{verbatim}

\subsubsection*{Level 9}
\begin{verbatim}
exact fun r : R => ⟨fun s : S => r, fun ns : ¬S => r⟩
\end{verbatim}

\subsubsection*{Question:}
Lambda calculus serves as a foundational framework for computation and formal logic. How can lambda calculus be used to reason about mnore advanced concepts, such as polymorphism?

\subsection{Week 7}

\subsubsection*{Question 1:}

Reduce the following lambda term:
\[ ((\lambda m.\lambda n.m\,n)\,(\lambda f.\lambda x.f\,(f\,x)))\,(\lambda f.\lambda x.f\,(f\,(f\,x))) \]

\textbf{Step 1:}
\begin{align*}
(\lambda m.\lambda n.m\,n)\,(\lambda f.\lambda x.f\,(f\,x)) &\rightarrow \lambda n.(\lambda f.\lambda x.f\,(f\,x))\,n
\end{align*}

\textbf{Step 2:}
\begin{align*}
\lambda n.(\lambda f.\lambda x.f\,(f\,x))\,n\,(\lambda f.\lambda x.f\,(f\,(f\,x))) &\rightarrow (\lambda f.\lambda x.f\,(f\,x))\,(\lambda f.\lambda x.f\,(f\,(f\,x)))
\end{align*}

\textbf{Step 3:} 
\[ \lambda x.((\lambda f.\lambda x.f\,(f\,(f\,x))))\,((\lambda f.\lambda x.f\,(f\,(f\,x)))\,x) \]

\begin{align*}
(\lambda f.\lambda x.f\,(f\,x))\,(\lambda f.\lambda x.f\,(f\,(f\,x))) &\rightarrow \lambda x.[(\lambda f.\lambda x.f\,(f\,(f\,x)))]\,((\lambda f.\lambda x.f\,(f\,(f\,x)))\,x)
\end{align*}

\textbf{Step 4:}
\begin{align*}
(\lambda f.\lambda x.f\,(f\,(f\,x)))\,x &\rightarrow \lambda x.x\,(x\,(x\,x))
\end{align*}

\begin{align*}
(\lambda f.\lambda x.f\,(f\,(f\,x)))\,x &\rightarrow x\,(x\,(x\,x))
\end{align*}

\textbf{Step 5:}
\[ \lambda x.((\lambda f.\lambda x.f\,(f\,(f\,x))))\,(x\,(x\,(x\,x))) \]

\textbf{Step 6:} 
\[ \lambda x.(x\,(x\,(x\,x)))\,(x\,(x\,(x\,x))\,x) \]

\textbf{Step 7:} 
\[ \lambda x.(x\,(x\,(x\,x)))\,(x\,(x\,(x\,x\,x))) \]

the function is applying $x$ six times, which corresponds to the Church numeral 6.


\subsubsection*{Question 2:}

Explain what function on natural numbers $\lambda m.\lambda n.m\,n$ implements.

\subsection*{Answer:}
The lambda term $\lambda m.\lambda n.m\,n$ represents the multiplication operation in Church numerals.

\textbf{Explanation:}
In Church encoding, natural numbers are represented as repeated applications of a function. The numeral $n$ is encoded as $\lambda f.\lambda x.f^n\,x$. The term $m\,n$ applies the function $m$ to $n$. Since $m$ represents applying a function $f$ $m$ times, and $n$ represents applying $f$ $n$ times, their composition results in $m \times n$ applications of $f$. Therefore, $\lambda m.\lambda n.m\,n$ takes two Church numerals and returns their product.

\subsubsection*{Question for Discord:}
Is it possible to represent data structures like pairs or lists using Church numerals and lambda calculus? If so, how do these representations work?

\subsection{Week 8/9}


\textbf{Question 2.}

The expression \texttt{a b c d} reduces to \texttt{(((a b) c) d)} becuase of the way lambda calculus handles application. Application is left associative. The function \texttt{a} is applied to \texttt{b}, then the result is applied to \texttt{c}, and finally, the result is applied to \texttt{d}. This can be written as:

\[
 (((a \, b) \, c) \, d)
\]

\textbf{Question 3.}

In lambda calculus, capture-avoiding substitution prevents bound variable names from unintentionally clashing with free variables. This is implemented in the \texttt{substitute} function, where:

- **Variables**: If a variable matches the name being substituted, it is replaced with the new expression; otherwise, it remains unchanged.

- **Lambda Abstractions**: If the bound variable matches the name being substituted, substitution is skipped for that scope. If not, a fresh name is generated using the \texttt{NameGenerator} to avoid conflicts, and substitution proceeds recursively.

- **Applications**: Substitution occurs recursively in both the function and argument.

\textbf{Example 1 (No Name Conflict)}:
\begin{verbatim}
(\x. \y. x) z
\end{verbatim}
Reduces to: 
\[
\lambda y. z
\]

\textbf{Example 2 (Name Conflict)}:
\begin{verbatim}
(\x. \y. x) y
\end{verbatim}
Reduces to: 
\[
\lambda \text{Var1}. y
\]
where \texttt{Var1} is a fresh variable generated to avoid conflict with \texttt{y}.


\textbf{Question 4.}
No, not all computations in lambda calculus reduce to normal form, and there are situations where evaluation may not terminate or result in a normal form.

Infinite Loops (Divergence): An expression like the following doesn't have a normal form:

\begin{verbatim}
(\x. x x) (\x. x x)
\end{verbatim}

\textbf{Question 5.}
Smallest lambda expression that does not reduce to normal form is known as the Omega combinator

\begin{verbatim}
  (\x. x x) (\x. x x)
\end{verbatim}

\textbf{Question 7.}

\begin{verbatim}
((\m.\n. m n) (\f.\x. f (f x))) (\f.\x. f (f (f x)))
((\Var1. (\f.\x. f (f x)) Var1) ) (\f.\x. f (f (f x)))
(\Var2. (\f.\x. f (f (f x))) ((\f.\x. f (f (f x))) Var2))
\end{verbatim}

\textbf{Question 8.}

\begin{verbatim}
  12: eval (((\m.(\n.(m n))) (\f.(\x.(f (f x))))) (\f.(\x.(f x))))
    39: eval ((\m.(\n.(m n))) (\f.(\x.(f (f x)))))
      39: eval (\m.(\n.(m n)))
      44: substitute ((\n.(m n)), 'm', (\f.(\x.(f (f x)))))
      45: eval (\Var1.((\f.(\x.(f (f x)))) Var1))
    44: substitute (((\f.(\x.(f (f x)))) Var1), 'Var1', (\f.(\x.(f (f (f x))))))
    45: eval ((\Var2.(\Var4.(Var2 (Var2 Var4)))) (\f.(\x.(f (f (f x))))))
      39: eval (\Var2.(\Var4.(Var2 (Var2 Var4))))
      44: substitute ((\Var4.(Var2 (Var2 Var4))), 'Var2', (\f.(\x.(f (f (f x))))))
      45: eval (\Var5.((\f.(\x.(f (f (f x))))) ((\f.(\x.(f (f (f x))))) Var5)))
\end{verbatim}


\textbf{Question on Discord:} -how does left-associative application impact the sequence and structure of reduction steps when evaluating complex expressions

\subsection{Week 10}


\begin{enumerate}
    \item \textbf{Most challenging aspect:} The biggest challenge was managing substitution in a way that avoided unintended variable capture, especially within deeply nested expressions. In the old `evaluate` function, substitution was only partially recursive, which sometimes led to incorrect evaluations when multiple applications or nested abstractions were involved. Ensuring that each function application and lambda abstraction was fully reduced required a more comprehensive approach, where both sides of applications were evaluated recursively and abstractions were reduced in sequence. This transformation was complex but essential for producing a reliable evaluation function.

    \item \textbf{Key insight for Assignment 3:} The key insight emerged from recognizing that the old `evaluate` function only partially evaluated applications by handling a single argument at a time. In the new version, `evaluate` was enhanced to handle both sides of a function application (`app`) recursively, which allowed for a more complete evaluation. Furthermore, an additional case was added to evaluate lambda abstractions (`lam`) by reducing their bodies, creating a more robust and normalized output. This recursive depth and abstraction evaluation were critical in managing nested applications and ensuring proper reduction to normal form.

    \item \textbf{Most interesting takeaway:} One of the most interesting takeaways from this assignment was understanding the full recursive structure required for accurate lambda calculus evaluation.
\end{enumerate}

\noindent \textbf{Question on Discord:} In what ways do modern interpreters and compilers handle complex recursive evaluations to ensure expressions reduce accurately


\subsection{Week 11}

\subsubsection*{Question 1: A = \{\}}

Since the set A is empty, there is nothing to draw for this ARS.

Terminating: Yes 

Confluent: Yes

Unique Normal Forms: Yes

\subsubsection*{Question 2: A = \{a\} and R = \{\}}

\[
\begin{tikzcd}
a
\end{tikzcd}
\]

Terminating: Yes

Confluent: Yes 

Unique Normal Forms: Yes

\subsubsection*{Question 3: A = \{a\} and R = \{a, a\}}

\[
\begin{tikzcd}
a \arrow[loop right]{}
\end{tikzcd}
\]

Terminating: No

Confluent: Yes 

Unique Normal Forms: No

\subsubsection*{Question 4: A = \{a, b, c\} and R = \{(a, b), (a, c)\}}

\[
\begin{tikzcd}
& a \arrow[ld] \arrow[rd] & \\
b & & c
\end{tikzcd}
\]

Terminating: Yes

Confluent: No

Unique Normal Forms: No

\subsubsection*{Question 5: A = \{a, b\} and R = \{(a, a), (a, b)\}}

\[
\begin{tikzcd}
a \arrow[loop above]{} \arrow[r] & b
\end{tikzcd}
\]

Terminating: No

Confluent: Yes

Unique Normal Forms: Yes

\subsubsection*{Question 6: A = \{a, b, c\} and R = \{(a, b), (b, b), (a, c)\}}

\[
\begin{tikzcd}
& a \arrow[ld] \arrow[rd] & \\
b \arrow[loop left]{} & & c
\end{tikzcd}
\]

Terminating: No

Confluent: No

Unique Normal Forms: No

\subsubsection*{Question 7: A = \{a, b, c\} and R = \{(a, b), (b, b), (a, c), (c, c)\}}

\[
\begin{tikzcd}
& a \arrow[ld] \arrow[rd] & \\
b \arrow[loop left]{} & & c \arrow[loop right]{}
\end{tikzcd}
\]

Terminating: No

Confluent: No

Unique Normal Forms: No

\subsubsection*{Homework: Try to find an example of an ARS for each of the possible 8 combinations. Draw pictures of these examples.}

\begin{center}
  \begin{tabular}{|c|c|c|c|c|}
  \hline
  \textbf{Confluent} & \textbf{Terminating} & \textbf{Has UNF} & \textbf{Example} \\
  \hline
  True & True & True & 
  \begin{tikzcd}
  a \arrow[r] & b
  \end{tikzcd} \\
  \hline
  True & True & False & N/A (Impossible) \\
  \hline
  True & False & True & 
  \begin{tikzcd}
  a \arrow[r] & b \arrow[loop right]{}
  \end{tikzcd} \\
  \hline
  True & False & False & 
  \begin{tikzcd}
  a \arrow[loop right]{}
  \end{tikzcd} \\
  \hline
  False & True & True & N/A \\
  \hline
  False & True & False & 
  \begin{tikzcd}
  & a \arrow[ld] \arrow[rd] & \\
  b & & c
  \end{tikzcd} \\
  \hline
  False & False & True & N/A \\
  \hline
  False & False & False & 
  \begin{tikzcd}
  a \arrow[r] & b \arrow[r] & c \arrow[ll, bend left]
  \end{tikzcd} \\
  \hline
  \end{tabular}
  \end{center}

\textbf{Question on Discord:} Why is it impossible to have an ARS that is both Confluent and Terminating but does not have Unique Normal Forms?


\subsection{Week 12} 

\text{Starting Point:}
\[
\text{let rec } \text{fact} = \lambda n. \text{if } n = 0 \text{ then } 1 \text{ else } n \times \text{fact}(n - 1) \text{ in } \text{fact } 3
\]

\text{Step 1: Expanding } \text{let rec}
\[
\text{let } \text{fact} = (\text{fix} \ (\lambda \text{fact}. \lambda n. \text{if } n = 0 \text{ then } 1 \text{ else } n \times \text{fact}(n - 1))) \text{ in } \text{fact } 3
\]

\text{Step 2: Expanding } \text{let}
\[
(\lambda \text{fact}. \text{fact } 3) (\text{fix} \ (\lambda \text{fact}. \lambda n. \text{if } n = 0 \text{ then } 1 \text{ else } n \times \text{fact}(n - 1)))
\]

\text{Step 3: Applying } \text{fix}
\[
(\text{fix} \ (\lambda \text{fact}. \lambda n. \text{if } n = 0 \text{ then } 1 \text{ else } n \times \text{fact}(n - 1))) \ 3
\]
\[
= (\lambda \text{fact}. \lambda n. \text{if } n = 0 \text{ then } 1 \text{ else } n \times \text{fact}(n - 1)) (\text{fix} \ (\lambda \text{fact}. \lambda n. \text{if } n = 0 \text{ then } 1 \text{ else } n \times \text{fact}(n - 1))) \ 3
\]

\text{Step 4: Applying the function}
\[
(\lambda n. \text{if } n = 0 \text{ then } 1 \text{ else } n \times (\text{fix} \ (\lambda \text{fact}. \lambda n. \text{if } n = 0 \text{ then } 1 \text{ else } n \times \text{fact}(n - 1))) (n - 1)) \ 3
\]

\text{Step 5: Computing } \text{if } n = 0
\[
\text{if } 3 = 0 \text{ then } 1 \text{ else } 3 \times (\text{fix} \ (\lambda \text{fact}. \lambda n. \text{if } n = 0 \text{ then } 1 \text{ else } n \times \text{fact}(n - 1))) (3 - 1)
\]
\[
= 3 \times (\text{fix} \ (\lambda \text{fact}. \lambda n. \text{if } n = 0 \text{ then } 1 \text{ else } n \times \text{fact}(n - 1))) \ 2
\]

\text{Step 6: Applying } \text{fix} \text{ again}
\[
3 \times (\lambda \text{fact}. \lambda n. \text{if } n = 0 \text{ then } 1 \text{ else } n \times \text{fact}(n - 1)) (\text{fix} \ (\lambda \text{fact}. \lambda n. \text{if } n = 0 \text{ then } 1 \text{ else } n \times \text{fact}(n - 1))) \ 2
\]

\text{Step 7: Applying the function}
\[
3 \times (\lambda n. \text{if } n = 0 \text{ then } 1 \text{ else } n \times (\text{fix} \ (\lambda \text{fact}. \lambda n. \text{if } n = 0 \text{ then } 1 \text{ else } n \times \text{fact}(n - 1))) (n - 1)) \ 2
\]

\text{Step 8: Computing } \text{if } n = 0 \text{ for } n = 2
\[
3 \times (\text{if } 2 = 0 \text{ then } 1 \text{ else } 2 \times (\text{fix} \ (\lambda \text{fact}. \lambda n. \text{if } n = 0 \text{ then } 1 \text{ else } n \times \text{fact}(n - 1))) (2 - 1))
\]
\[
= 3 \times (2 \times (\text{fix} \ (\lambda \text{fact}. \lambda n. \text{if } n = 0 \text{ then } 1 \text{ else } n \times \text{fact}(n - 1))) \ 1)
\]

\text{Step 9: Applying } \text{fix} \text{ one more time}
\[
3 \times 2 \times (\lambda \text{fact}. \lambda n. \text{if } n = 0 \text{ then } 1 \text{ else } n \times \text{fact}(n - 1)) (\text{fix} \ (\lambda \text{fact}. \lambda n. \text{if } n = 0 \text{ then } 1 \text{ else } n \times \text{fact}(n - 1))) \ 1
\]

\text{Step 10: Applying the function}
\[
3 \times 2 \times (\lambda n. \text{if } n = 0 \text{ then } 1 \text{ else } n \times (\text{fix} \ (\lambda \text{fact}. \lambda n. \text{if } n = 0 \text{ then } 1 \text{ else } n \times \text{fact}(n - 1))) (n - 1)) \ 1
\]

\text{Step 11: Computing } \text{if } n = 0 \text{ for } n = 1
\[
3 \times 2 \times (\text{if } 1 = 0 \text{ then } 1 \text{ else } 1 \times (\text{fix} \ (\lambda \text{fact}. \lambda n. \text{if } n = 0 \text{ then } 1 \text{ else } n \times \text{fact}(n - 1))) (1 - 1))
\]
\[
= 3 \times 2 \times (1 \times (\text{fix} \ (\lambda \text{fact}. \lambda n. \text{if } n = 0 \text{ then } 1 \text{ else } n \times \text{fact}(n - 1))) \ 0)
\]

\text{Step 12: Applying } \text{fix} \text{ again}
\[
3 \times 2 \times 1 \times (\lambda \text{fact}. \lambda n. \text{if } n = 0 \text{ then } 1 \text{ else } n \times \text{fact}(n - 1)) (\text{fix} \ (\lambda \text{fact}. \lambda n. \text{if } n = 0 \text{ then } 1 \text{ else } n \times \text{fact}(n - 1))) \ 0
\]

\text{Step 13: Applying the function}
\[
3 \times 2 \times 1 \times (\lambda n. \text{if } n = 0 \text{ then } 1 \text{ else } n \times (\text{fix} \ (\lambda \text{fact}. \lambda n. \text{if } n = 0 \text{ then } 1 \text{ else } n \times \text{fact}(n - 1))) (n - 1)) \ 0
\]

\text{Step 14: Computing } \text{if } n = 0 \text{ for } n = 0
\[
3 \times 2 \times 1 \times (\text{if } 0 = 0 \text{ then } 1 \text{ else } 0 \times (\text{fix} \ (\lambda \text{fact}. \lambda n. \text{if } n = 0 \text{ then } 1 \text{ else } n \times \text{fact}(n - 1))) (0 - 1))
\]
\[
= 3 \times 2 \times 1 \times 1 \quad (\text{since } 0 = 0, \text{ take the } \text{then} \text{ branch})
\]

\text{Final Computation:}
\[
3 \times 2 \times 1 \times 1 = 6
\]

\text{The result is } 6.

\subsection*{Question for discord:}
are there specific cases where using recursion in lambda calculus can lead to inefficiencies that an iterative approach could solve more cleanly?



\subsection{Week 13}
\textbf{Excercise 8:}  
Consider the rewrite rules:  
\[
ab \to cc, \quad ac \to bb, \quad bc \to aa
\]
plus rules saying that the order of letters does not matter. Starting from \( 15a \), \( 14b \), and \( 13c \), is it possible to reach a configuration in which there are only \( a \)'s, only \( b \)'s, or only \( c \)'s?

\subsection*{Solution}

1. \textbf{Invariant:}  
   A helpful approach is to use an invariant that remains unchanged under the rewrite rules. By analyzing the given rules:  
   \begin{itemize}
       \item \( ab \to cc \): reduces \( a + b \), increases \( c \) by \( +2 \).
       \item \( ac \to bb \): reduces \( a + c \), increases \( b \) by \( +2 \).
       \item \( bc \to aa \): reduces \( b + c \), increases \( a \) by \( +2 \).
   \end{itemize}
   The total number of letters \( f(a, b, c) = a + b + c \) remains constant under all rules.

   Initial value:
   \[
   f(15, 14, 13) = 15 + 14 + 13 = 42.
   \]

   Therefore, \( 42 \) letters must be preserved at all times.

2. \textbf{Parity Analysis:}  
   Under each rule:
   \begin{itemize}
       \item \( ab \to cc \): The count of \( a + b + c \) remains unchanged.
       \item \( ac \to bb \): Similarly, parity does not change.
       \item \( bc \to aa \): Again, parity does not change.
   \end{itemize}

   Since the initial counts of \( a \), \( b \), and \( c \) are \( 15, 14, 13 \) respectively (all odd or even), parity is preserved.

To reach a state where only one type of letter (\( a \), \( b \), or \( c \)) remains:
\begin{itemize}
    \item At least one letter type must dominate entirely while the other two reduce to zero.
    \item However, the rules \( ab \to cc \), \( ac \to bb \), and \( bc \to aa \) always generate new letters of the other two types.
\end{itemize}
It is impossible to reach a state containing only \( a \)'s, \( b \)'s, or \( c \)'s because the system keeps ``mixing'' types.

---

\subsubsection*{Conclusion}
Starting from \( 15a \), \( 14b \), \( 13c \), it is not possible to reduce the configuration to only \( a \)'s, \( b \)'s, or \( c \)'s. The invariant \( f(a, b, c) = 42 \) and the parity analysis confirm this result.




\subsubsection*{Question for discord:}
How could introducing invariants based on patterns (e.g., tracking the balance of transformations across rules) lead to unexpected equivalence classes?
  

\section{Lessons from the Assignments}

The assignments helped me see how ideas from computer science theory connect to real world uses. 
We used Lean to prove basic math rules that we often take for granted. Including: Reflexivity
(showing that something equals itself), Commutativity (showing that the order doesn't matter, 
like 2+3 = 3+2), andAssociativity (showing that grouping doesn't matter, like (1+2)+3 = 1+(2+3))
Lean has special commands that help prove these rules step by step. For instance, when we need 
to prove that addition is associative, we used a method called induction - which is like proving 
that if something works for the first number, then showing that if it works for one number, it 
must work for the next one too. Lean helped by automatically checking each logical step. This 
work isn't just theoretical - it has practical value. When we build important computer systems 
that need to be mathematically perfect (like security systems or programs that check other 
mathematical proofs), tools like Lean help make sure everything works exactly as it should.

The assignments taught me about lambda calculus. I think of it as a very simple programming language 
that's become the foundation for many modern programming languages. This work helped me understand 
how modern programming languages handle variables and functions, especially in languages like Haskell 
and Scala where functions can be passed around like any other data. We added math operations and if 
statements to our simple system. This showed how it can be built upon to create a full programming 
language. It's like seeing how you can start with just a few bricks and build something quite complex. 
Another valuable lesson came from evaluating Church numerals. This exercise not only showcased how 
numbers and arithmetic operations can be represented functionally but also highlighted the elegance 
and power of recursion. Also non-terminating evaluations—became apparent. This emphasized the 
importance of carefully managing recursion in programming and algorithm design.

The assignments reinforced the importance of invariants in reasoning about system behavior. One notable 
exercise involved examining rewrite rules for strings while preserving the total count of characters. 
This reasoning provided an approach to solving problems and demonstrated how to detect impossibilities, 
such as reducing a mixed system of letters to a single type. This principle is broadly applicable in 
areas like database management, where maintaining data integrity is critical, and in distributed systems,
where consistent state synchronization is paramount. It's like having an anchor point that helps you check 
if things are going wrong and proves whether certain changes are even possible.

I learned about using Lean to helps check if mathematical proofs are correct. 
Instead of having humans check every step of a proof (which can be time consuming and error prone), 
Lean does this automatically. Its like having a super-thorough proofreader who never gets tired and 
never misses a mistake. This is especially important when we're building systems where mistakes could be 
dangerous or very expensive, like: Software that controls airplanes or Systems that handle large financial transactions.
The cool thing about Lean is that it makes this kind of careful checking more: Doable (because computers can check 
things much faster than humans), Reliable (because computers don't get tired or distracted), and Scalable (because once 
you've proven something, you can use that proof as a building block for bigger proofs) It's like turning mathematical 
reasoning into a language that computers can understand and verify, which gives us a powerful way to tackle really 
complex problems while being sure our solutions are correct.


A particularly challenging but rewarding part of the assignments was managing nested expressions and achieving 
normal form in lambda calculus. The recursive evaluation process required careful handling of abstractions, 
applications, and substitutions to avoid unintended variable capture. Enhancing the evaluate function to 
handle deeply nested expressions underscored the recursive nature of computation and the importance of 
designing robust algorithms that handle edge cases effectively.

We learned about a special tool in programming called the fixed point combinator that lets us write functions that 
can call themselves without using normal loops. While this sounds complex, it's actually a clever way to make recursive 
functions work in a very basic programming system. This might seem like a roundabout way to do something simple, but it 
shows something really powerful: you can create complex calculations using nothing but functions that transform other 
functions. It's like building a complex machine using only simple parts. This matters because: It helps us understand how 
programming languages like Haskell work under the hood, it shows how abstract math ideas can solve real programming problems, and
it demonstrates how to build complex things from very simple building blocks

We learned about turning text that humans write into a structure that computers can more easily work with. This process is called parsing, 
and it's like taking apart a sentence to understand its grammar, but for computer code. When you write `1 + 2 * 3`, a human knows that 
multiplication should happen before addition. But a computer needs this spelled out more clearly. So we turn it into a tree-like structure 
that shows exactly what should happen in what order. It's like adding parentheses to make everything crystal clear: `1 + (2 * 3)`.
We used a tool called Lark in Python that helps build these parsers. Lark is like having a set of rules (called a grammar) that tells the computer: 
how to recognize numbers, which operations should happen first and how to handle more complex expressions. It's like being a translator between 
human language and computer language, where every little detail needs to be precisely defined. This precision is crucial because computers 
need absolutely clear instructions to work correctly.


We did an interesting assignment using LLMs to help us research the history and development of programming languages. 
We Started with a question we were curious about in programming languages, used the LLM to help us explore that topic deeper
, found connections between different areas of programming language research and discovered who the important researchers were and 
what they contributed. I learned to carefully check and edit what the llm suggests and how llms can help you find connections and patterns. 
This homework taught me two main things: How different areas of programming languages connect and influence each other
and how ow to use models effectively as a research tool. It's like having a smart brainstorming partner who can help you explore ideas and 
make connections, but occasionally likes to make stuff up. 


\section{Conclusion}\label{conclusion}


This course has taught me how theoretical computer science ideas connect to real world programming.
First, I got hands-on experience with different tools and concepts:
Using Lean to check mathematical proofs automatically, which showed me how computers can help ensure critical systems work correctly,
learning about lambda calculus, which is like a simple blueprint that modern programming languages are built on, 
understanding how computers read and understand code by breaking it down into a structure they can process and more. 

The part of the course focused on turning human written into something computers can understand taught me why programming languages 
need to be designed so carefully - it's like creating a new language that needs to be clear to both humans and machines.

The biggest takeaway was seeing how abstract math and logic aren't just theoretical - they help solve real problems. 
The course gave me both knowledge and new ways to tackle complex challenges. It's like learning to build things with both the basic 
building blocks (the theory) and modern power tools (the practical applications), and understanding why both are important.


\begin{thebibliography}{99}
\bibitem[BLA], Author, Title, Publisher, Year.
\end{thebibliography}
\end{document}
