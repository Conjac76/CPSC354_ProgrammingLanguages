\documentclass{article}
\usepackage{amsmath}
\usepackage{hyperref}
\usepackage{tocloft}

\title{Report}
\author{Connor Jacobs}
\date{}

\begin{document}

% Title Page
\maketitle
\newpage

% Table of Contents
\tableofcontents
\newpage

\section*{Week 1}
\addcontentsline{toc}{section}{Week 1}


\subsubsection*{Mathematical Proof}
\addcontentsline{toc}{subsubsection}{Mathematical Proof}

Proving that $37x + q = 37x + q$ demonstrates the reflexivity property of equality. Reflexivity states that any mathematical expression is equal to itself. In this case, the expression $37x + q$ is compared to itself, and it is immediately clear by the reflexivity of equality that this statement is true.

\subsubsection*{Proof in Lean}
\addcontentsline{toc}{subsubsection}{Proof in Lean}

In Lean, the theorem $37x + q = 37x + q$ is proven using the `rfl` tactic. The `rfl` tactic in Lean stands for "reflexivity" and handles proofs of the form $X = X$. When `rfl` is executed, Lean verifies that both sides of the equation are equal and the proof is complete.

The command to execute in Lean is simply:
\begin{verbatim}
rfl
\end{verbatim}

\subsubsection*{Connection Between Lean and Mathematics}
\addcontentsline{toc}{subsubsection}{Connection Between Lean and Mathematics}

In mathematics, we rely on the axiom of reflexivity to assert that $37x + q = 37x + q$. Likewise, in Lean, the `rfl` tactic automates this process by invoking the same principle. The use of `rfl` in Lean serves as a direct representation of reflexivity in mathematical logic, providing an automated and formalized way to conclude that an expression is equal to itself.

Thus, both in Lean and in traditional mathematics, the proof of $37x + q = 37x + q$ is an application of the same fundamental concept: the reflexivity of equality.

\end{document}
