\documentclass{article}
\usepackage{graphicx} % Required for inserting images
\usepackage{amsmath}  % Required for using the \text command and align environment
\usepackage[T1]{fontenc} % Better font encoding to handle text more robustly
\usepackage{hyperref} % Required for clickable links in the table of contents

\title{Report}
\author{Connor Jacobs}
\date{August 2024}
\begin{document}

\maketitle

\newpage
\tableofcontents % Generates the table of contents

\newpage
\section{Week 1, Day 1: NNG Tutorial World}
In this report, we will explore the concept of numbers as defined in Lean, specifically focusing on the level titled "The Birth of Number" from the project Tut. World.

\subsection{The Birth of Number}
Numbers in Lean are defined by two fundamental rules:
\begin{itemize}
    \item \textbf{0 is a number.}
    \item \textbf{If \( n \) is a number, then the successor \( \text{succ } n \) of \( n \) is a number.}
\end{itemize}

The successor of \( n \) is the number that comes immediately after \( n \). This concept allows us to start counting and define basic numbers.

\subsubsection{Counting to Four}
Starting from \( 0 \), we can define the first few numbers as follows:
\begin{align*}
    1 &= \text{succ } 0, \\
    2 &= \text{succ } 1, \\
    3 &= \text{succ } 2, \\
    4 &= \text{succ } 3.
\end{align*}

\subsubsection{Proof: \( 2 = \text{succ } 1 \)}
The proof that \( 2 = \text{succ } 1 \) is provided in Lean under the lemma named \texttt{two\_eq\_succ\_one}. Let's prove that \( 2 \) is indeed the number after the number after zero.

\paragraph{Step-by-Step Proof}
To begin, we start by rewriting \( 2 \) using the lemma \texttt{two\_eq\_succ\_one}:
\begin{verbatim}
rw [two_eq_succ_one]
\end{verbatim}

Next, we rewrite \( 1 \) as \( \text{succ } 0 \) using:
\begin{verbatim}
rw [one_eq_succ_zero]
\end{verbatim}

Finally, we verify that the equation holds by using the reflexivity command:
\begin{verbatim}
rfl
\end{verbatim}

\subsubsection{Alternative Approach}
The proof can also be completed in a more concise manner by combining the rewrite steps:
\begin{verbatim}
rw [two_eq_succ_one, one_eq_succ_zero]
rfl
\end{verbatim}
\end{document}
